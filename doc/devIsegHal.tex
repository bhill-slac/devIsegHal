\documentclass[a4paper,10pt]{scrartcl}
%
\usepackage[english]{babel}
\usepackage{lmodern}
\usepackage{textcomp}
\usepackage{amssymb}
\usepackage{amsmath}
\usepackage{color, xcolor, colortbl}
\usepackage{listings}
\usepackage{etoolbox}
\usepackage[colorlinks,linkcolor = black,citecolor = black] {hyperref}
\usepackage[final]{pdfpages}
\usepackage[a4paper]{geometry}
\usepackage[automark]{scrpage2}
\usepackage{upquote}
\pagestyle{scrheadings}
\clearscrheadfoot
%
\setlength{\voffset}{-1.5cm}
\setlength{\oddsidemargin}{-1.8cm}
\setlength{\evensidemargin}{-1.8cm}
\setlength{\textheight}{24cm}
\setlength{\textwidth}{19.5cm}
\setlength{\unitlength}{1.0cm}
%
\newbool{forLMD}
\newbool{cross}
\setbool{forLMD}{false}
\setbool{cross}{false}
%
\definecolor{darkgray}{rgb}{0.8,0.8,0.8}
\definecolor{darkgreen}{rgb}{0,.5,0}
\definecolor{lightgray}{rgb}{0.95,0.95,0.95}
%
\lstset{ %
  language=C,
  basicstyle=\small\ttfamily,       % the size of the fonts that are used for the code
  keywordstyle=\color{blue}\bfseries,
  stringstyle=\color{darkgreen},%\ttfamily,
  commentstyle=\color{red},
  numbers=left,                   % where to put the line-numbers
  numberstyle=\tiny,
  stepnumber=5,
  numbersep=5pt,
  backgroundcolor=\color{lightgray},   % choose the background color. You must add \usepackage{color}
  showspaces=false,               % show spaces adding particular underscores
  showstringspaces=false,         % underline spaces within strings
  showtabs=false,                 % show tabs within strings adding particular underscores
  frame=single,                   % adds a frame around the code
  tabsize=2,                      % sets default tabsize to 2 spaces
  captionpos=b,                   % sets the caption-position to bottom
  breaklines=true,                % sets automatic line breaking
  breakatwhitespace=false,        % sets if automatic breaks should only happen at whitespace
  title=\lstname,                 % show the filename of files included with \lstinputlisting;
                                  % also try caption instead of title
  columns=flexible,
  keepspaces=true,
  morekeywords={record, field}
}
%
\ihead[]{}                      %rechte Kopfzeile
\chead[]{}                      %mittlere Kopfzeile
\ohead[\pagemark]{\pagemark}    %linke Kopfzeile (Seitenzahl)
\setheadsepline{.4pt}           %Trennlinie Kopfzeile-Text
%
\title{devIsegHal\\EPICS Device Support for\\ iseg Hardware Abstraction Layer}
\author{F.~Feldbauer $<$feldbaue@kph.uni-mainz.de$>$}
\date{\today}
%
\begin{document}
%
% * * * * * * * * * * * * * * * * * * * * * * * * * * * * * * * * * * * * * *
%
\maketitle
\tableofcontents 
%
% * * * * * * * * * * * * * * * * * * * * * * * * * * * * * * * * * * * * * *
%
\section{Introduciton}
The isegHAL library offers a string based application interface - API. The data
collection from the iseg high voltage modules is done in background. All
communication hand shake is handled by the isegHAL.\\
This module offers EPICS device support routines to use the isegHAL library
within your EPICS applicaitons.

\section{Usage}
\subsection{Connect to an interface with the isegHalServer}
Before loading any records, devIsegHal has to connect to an interface
with the isegHalServer daemon. This is done via the IOC shell command
\begin{lstlisting}
isegHalConnect( "NAME", "INTERFACE" )
\end{lstlisting}
NAME is a user defined name, which is internally used to address this interface
while INTERFACE is the actual name of the hardware interface from your operating system
(e.g. "can0" for a CAN interface)

\subsection{Records}
To make a record use devIsegHal, set its \lstinline|DTYP| field to "isegHAL".
The \lstinline|INP| or \lstinline|OUT| link has the form "@OBJECT IF".
Here OBJECT is a fully qualified object string for the item values
provided by the isegHalServer and IF is the name of the interface as used with the 
\lstinline|isegHalConnect| command mentioned above.\\
isegHAL provides its own timestamp of the last change of a value (\lstinline|timeStampLastChanged|).
To use this timestamp as timestamp of the record, the \lstinline|TSE| field has to be set to $-2$\\
Example:
\begin{lstlisting}
record( ai, "ISEG:0:0:2:VoltageMeasure" ) {
  field( DTYP, "isegHAL" )
  field( INP,  "@0.0.2.VoltageMeasure can0" )
  field( TSE,  "-2" )
}
\end{lstlisting}
If the \lstinline|EGU| field is not set in the database, the unit-value from the
corresponding IsegItemProperty is copied into this field during initialization.\\

\section{Asynchronous Handling}
It is possible that control parameters change during operation. For example, if a trip occures
the corresponding \lstinline|setON| bit in the channel control register will be set to 0.
These changes are monitored by devIsegHal through a polling thread.
Each output record (execpt for stringout records) is automatically registered to this thread
and their values are checked for updates on the isegHAL. If a value has changed
the \lstinline|VAL| field and timestamp of the record will be set to the new values.\\
Setting the \lstinline|SCAN| field of input records to \lstinline|I/O Intr| will also
register these records for the thread monitoring the values in isegHAL.\\
The thread goes through the list of registered records, checks each for an update, and
then waits for 5 seconds. This waiting time can be modified using the IOC Shell Commands (c.f.~\ref{iocCmd}).

\section{Supported Record Types}
\begin{tabular}{|l|c|}
\hline
Record type & isegDataType \\ \hline
ai/ao records & R4 \\
bi/bo records & BOOL \\
mbbiDirect records & UI1 \& UI4\\
longin/longout records & UI1 \& UI4\\
stringin/stringout records & STR\\
\hline
\end{tabular}

{\em Note: the maximum string length for stringin/out records is limited to 40 characters while the maximal length for the value of an IsegItemValue is 200.
Thus only the first 39 characters of the IsegItemValue are copied to record's VAL field (plus Null-Character for string termination).}

\section{IOC Shell Commands}\label{iocCmd}
Currently there is only on command callable from the IOC shell.
\begin{lstlisting}
devIsegHalSetOpt( "key", "value" )
\end{lstlisting}
\begin{tabular}{|l|l|c|}
\hline
Key & Meaning & Value \\ \hline
Intervall & Change the intervall of the polling thread & a value of 0 means no pause between two iterations of the list \\
LogLevel & Change log level of isegHalServer & see isegHal Manual \\
\hline
\end{tabular}

\end{document}

